\chapter{Conclusione}
In questo studio si è potuta verificare l'efficacia del 
federated learning ibrido. La condivisione di una piccola parte dei 
dataset locali può essere una strategia efficacie per affrontare le 
difficoltà poste dal federated learning.

I risultati dimostrano che condividere una parte dei dataset ha un 
effetto positivo apprezzabile sulle performance del modello. Sono stati 
considerati due dataset di due problemi fondamentalmente diversi -UCI HAR 
e FEMNIST- a confermare il fatto che i risultati non siano una particolarità
dello specifico problema o modello utilizzato. Il miglioramento di 
performance è stato particolarmente notevole nel FEMNIST che si è rivelato 
il dataset più complicato dei due. 

Tuttavia rimane importante bilanciare in modo appropriato la condivisione 
di dati, con la protezione della privacy ove necessario e informare 
l'utente della condivisione in atto, comportamento virtuoso anche in 
assenza di legislazione specifica come il GDPR.

I due dataset diversi sono stati provati con diverse configurazioni: il 
modello è rimasto lo stesso, ma è stato allenato con diversi algoritmi
di ottimizzazione, diversi gradi di condivisione dei dataset e diverse 
tecniche per la condivisione. I risultati mostrano che AdamW è un 
algoritmo tendenzialmente più instabile, che miglioramenti delle 
performance apprezzabili possono essere ottenuti con una condivisione 
di non più del 20\% e che condividere pochi dati di ogni client è 
permette un miglioramento più sicuro sulle performance rispetto al 
selezionare alcuni client specifici.

In conclusione, questa tesi contribuisce alla crescente mole di letteratura
nel machine learning, in generale, e federated learning, nello specifico, 
evidenziando il potenziale promettente di questo modello di apprendimento
automatico. Lavori futuri potrebbero esplorare iperparametri diversi e più
ottimali per provare ad allenare modelli \textit{compute optimal} oppure 
studiare benchmark più complessi per performance \textit{state of the art}.
