\chapter{Introduzione}
La rapida proliferazione di dispositivi connessi sulla rete, insieme 
ad una maggiore sensibilità diffusa nei confronti della privacy 
hanno fatto sì che si rendessero necessari nuovi modelli di machine 
learning che tenessero in considerazione i requisiti di contesti 
privacy oriented. Il Federated learning (FL) è nuovo approccio che 
risolve questo problema permettendo a diversi client (ad esempio
telefoni, dispositivi IoT o server edge) distribuiti per il mondo di 
allenare collaborativamente uno stesso modello di machine learning. La 
privacy dei dati è preservata mantenendo i dati all'origine, sui 
client stessi che li producono, e portando invece il modello da 
allenare ad essi.

Nonostante le sue potenzialità, il federated learning ha comunque delle
difficoltà, prima fra tutte il fatto che tipicamente i dati locali 
dei client sono estremamente eterogenei tra loro, peggiorando le 
performance del modello allenato. Una delle assunzioni tipiche per le 
tecniche dei machine learning più diffuse è che il dataset utilizzato 
per l'apprendimento sia costituito da sample IID (Indipendenti e 
Identicamente Distribuiti), fatto che tipicamente non si verifica in 
contesti federati.

Una delle possibili strategie per migliorare le capacità di un modello 
allenato in modo federato è quella di costruire un dataset globale, 
composto da un sottoinsieme dei dataset dei client che viene condiviso 
e reso pubblico, il federated learning ibrido. Questa tesi studia la 
variazione di performance di un modello sotto diversi gradi di 
ibridazione. Vengono presi in considerazione due dataset diversi nel 
dominio dell'image classification e del human activity recognition 
-FEMNIST e UCI HAR- per vedere come si comporta il federated 
learning sotto diverse modalità. Il FEMNIST (Federated Extended MNIST)
è un'estensione del celebre MNIST che include immagini sia di cifre 
che di numeri e i cui caratteri sono partizionati in base allo scrittore
originale; l'UCI HAR invece contiene misurazioni fatte 
da accellerometro e giroscopio di uno smartphone tenuto in vita da 30
persone diverse mentre compiono azioni diverse. Per studiare al meglio 
gli effetti di diverse strategie di condivisione dei dati, si è 
allenata la stessa rete neurale con diversi gradi di condisione dei 
dati (da 0\%, federato, a 100\%, centralizzato) e diversi metodi di 
condivisione dei dati. Per il FEMNIST è stata usata una CNN e per UCI 
HAR un MLP. Entrambi i modelli sono stati allenati sia con SGD che 
con AdamW, due algoritmi di ottimizzazione tipici nel Deep Learning.


Questo elaborato è strutturato nel modo seguente: il capitolo 2 
introduce il federated learning, diverse categorizzazioni e strategie, 
seguiti da una discussione di vantaggi e limitazioni; il capitolo 3 
discute nel dettaglio gli esperimenti, descrivendo i dataset e i 
modelli usati e spiegando le scelte fatte; il capitolo 4 illustra 
l'implementazione del progetto in codice python, facendo uso di Flower
~\cite{flowerai}, un framework per fare simulazioni o applicazioni
di federated learning che supporta 
sia PyTorch che Tensorflow e vari ambienti di 
sviluppo oltre python tra cui Android, iOS e C++; infine il capitolo 5
mostra i risultati degli esperimenti e discute anche possibili metodi 
per migliorare ulteriormente le performance del modello.
